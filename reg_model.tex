% Options for packages loaded elsewhere
\PassOptionsToPackage{unicode}{hyperref}
\PassOptionsToPackage{hyphens}{url}
%
\documentclass[
]{article}
\usepackage{amsmath,amssymb}
\usepackage{iftex}
\ifPDFTeX
  \usepackage[T1]{fontenc}
  \usepackage[utf8]{inputenc}
  \usepackage{textcomp} % provide euro and other symbols
\else % if luatex or xetex
  \usepackage{unicode-math} % this also loads fontspec
  \defaultfontfeatures{Scale=MatchLowercase}
  \defaultfontfeatures[\rmfamily]{Ligatures=TeX,Scale=1}
\fi
\usepackage{lmodern}
\ifPDFTeX\else
  % xetex/luatex font selection
\fi
% Use upquote if available, for straight quotes in verbatim environments
\IfFileExists{upquote.sty}{\usepackage{upquote}}{}
\IfFileExists{microtype.sty}{% use microtype if available
  \usepackage[]{microtype}
  \UseMicrotypeSet[protrusion]{basicmath} % disable protrusion for tt fonts
}{}
\makeatletter
\@ifundefined{KOMAClassName}{% if non-KOMA class
  \IfFileExists{parskip.sty}{%
    \usepackage{parskip}
  }{% else
    \setlength{\parindent}{0pt}
    \setlength{\parskip}{6pt plus 2pt minus 1pt}}
}{% if KOMA class
  \KOMAoptions{parskip=half}}
\makeatother
\usepackage{xcolor}
\usepackage[margin=1in]{geometry}
\usepackage{graphicx}
\makeatletter
\def\maxwidth{\ifdim\Gin@nat@width>\linewidth\linewidth\else\Gin@nat@width\fi}
\def\maxheight{\ifdim\Gin@nat@height>\textheight\textheight\else\Gin@nat@height\fi}
\makeatother
% Scale images if necessary, so that they will not overflow the page
% margins by default, and it is still possible to overwrite the defaults
% using explicit options in \includegraphics[width, height, ...]{}
\setkeys{Gin}{width=\maxwidth,height=\maxheight,keepaspectratio}
% Set default figure placement to htbp
\makeatletter
\def\fps@figure{htbp}
\makeatother
\setlength{\emergencystretch}{3em} % prevent overfull lines
\providecommand{\tightlist}{%
  \setlength{\itemsep}{0pt}\setlength{\parskip}{0pt}}
\setcounter{secnumdepth}{-\maxdimen} % remove section numbering
\ifLuaTeX
  \usepackage{selnolig}  % disable illegal ligatures
\fi
\usepackage{bookmark}
\IfFileExists{xurl.sty}{\usepackage{xurl}}{} % add URL line breaks if available
\urlstyle{same}
\hypersetup{
  pdftitle={Регрессионая модель},
  pdfauthor={Ядонист Роман},
  hidelinks,
  pdfcreator={LaTeX via pandoc}}

\title{Регрессионая модель}
\author{Ядонист Роман}
\date{}

\begin{document}
\maketitle

Загрузим данные

Добавим столбец с временем года

\begin{enumerate}
\def\labelenumi{\arabic{enumi}.}
\tightlist
\item
  Помтроим регрессионную модель, которая позволяет выяснить:
\end{enumerate}

\begin{itemize}
\tightlist
\item
  каким образом потребительская активность зависит от средней заработной
  платы в регионе, числа активных абонентов беспроводного наземного
  фиксированного доступа к сети Интернет и уровня безработицы населения.
\end{itemize}

\begin{verbatim}
## 
## Call:
## lm(formula = Индекс.ПА ~ `Среднемесячная з.п.` + 
##     `Число абонентов` + `Уровень безработицы` + 
##     Сезон, data = df)
## 
## Residuals:
##      Min       1Q   Median       3Q      Max 
## -25.5183  -4.1102  -0.0585   4.2416  27.3217 
## 
## Coefficients:
##                         Estimate Std. Error t value Pr(>|t|)    
## (Intercept)            5.874e+01  8.015e-01  73.285  < 2e-16 ***
## `Среднемесячная з.п.` -1.019e-05  1.087e-05  -0.937   0.3489    
## `Число абонентов`      5.516e-05  3.053e-05   1.807   0.0711 .  
## `Уровень безработицы` -2.463e-01  5.179e-02  -4.756 2.27e-06 ***
## Сезонзима              8.476e+00  6.035e-01  14.046  < 2e-16 ***
## Сезонлето              1.324e+01  6.019e-01  21.998  < 2e-16 ***
## Сезоносень             1.312e+01  6.018e-01  21.808  < 2e-16 ***
## ---
## Signif. codes:  0 '***' 0.001 '**' 0.01 '*' 0.05 '.' 0.1 ' ' 1
## 
## Residual standard error: 6.712 on 989 degrees of freedom
## Multiple R-squared:  0.4023, Adjusted R-squared:  0.3987 
## F-statistic:   111 on 6 and 989 DF,  p-value: < 2.2e-16
\end{verbatim}

\begin{itemize}
\tightlist
\item
  каким образом доля безналичных платежей в торговом обороте зависит от
  средней заработной платы в регионе, числа активных абонентов
  беспроводного наземного фиксированного доступа к сети Интернет и
  уровня безработицы населения.
\end{itemize}

\begin{verbatim}
## 
## Call:
## lm(formula = Индекс.БП ~ `Среднемесячная з.п.` + 
##     `Число абонентов` + `Уровень безработицы` + 
##     Сезон, data = df)
## 
## Residuals:
##      Min       1Q   Median       3Q      Max 
## -28.6934  -3.9745  -0.3483   4.3246  19.7348 
## 
## Coefficients:
##                         Estimate Std. Error t value Pr(>|t|)    
## (Intercept)            5.459e+01  7.853e-01  69.510  < 2e-16 ***
## `Среднемесячная з.п.`  1.370e-04  1.065e-05  12.865  < 2e-16 ***
## `Число абонентов`      8.419e-05  2.991e-05   2.815 0.004979 ** 
## `Уровень безработицы` -1.354e+00  5.075e-02 -26.672  < 2e-16 ***
## Сезонзима             -1.103e+00  5.913e-01  -1.865 0.062528 .  
## Сезонлето              1.471e-01  5.897e-01   0.249 0.803054    
## Сезоносень             2.135e+00  5.896e-01   3.620 0.000309 ***
## ---
## Signif. codes:  0 '***' 0.001 '**' 0.01 '*' 0.05 '.' 0.1 ' ' 1
## 
## Residual standard error: 6.577 on 989 degrees of freedom
## Multiple R-squared:  0.5448, Adjusted R-squared:  0.5421 
## F-statistic: 197.3 on 6 and 989 DF,  p-value: < 2.2e-16
\end{verbatim}

\begin{verbatim}
## 
## % Table created by stargazer v.5.2.3 by Marek Hlavac, Social Policy Institute. E-mail: marek.hlavac at gmail.com
## % Date and time: Mon, Aug 12, 2024 - 15:50:43
## \begin{table}[!htbp] \centering 
##   \caption{} 
##   \label{} 
## \begin{tabular}{@{\extracolsep{5pt}}lccccc} 
## \\[-1.8ex]\hline 
## \hline \\[-1.8ex] 
## Statistic & \multicolumn{1}{c}{N} & \multicolumn{1}{c}{Mean} & \multicolumn{1}{c}{St. Dev.} & \multicolumn{1}{c}{Min} & \multicolumn{1}{c}{Max} \\ 
## \hline \\[-1.8ex] 
## rating & 30 & 64.633 & 12.173 & 40 & 85 \\ 
## complaints & 30 & 66.600 & 13.315 & 37 & 90 \\ 
## privileges & 30 & 53.133 & 12.235 & 30 & 83 \\ 
## learning & 30 & 56.367 & 11.737 & 34 & 75 \\ 
## raises & 30 & 64.633 & 10.397 & 43 & 88 \\ 
## critical & 30 & 74.767 & 9.895 & 49 & 92 \\ 
## advance & 30 & 42.933 & 10.289 & 25 & 72 \\ 
## \hline \\[-1.8ex] 
## \end{tabular} 
## \end{table}
\end{verbatim}

\end{document}
